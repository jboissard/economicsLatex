%
%  untitled
%
%  Created by Johan Boissard [] on 2010-06-24.
%  Copyright (c) Johan Boissard. All rights reserved.
% hhh

\documentclass[a4paper] {scrartcl}
\usepackage[T1]{fontenc}
\usepackage[utf8]{inputenc}
\usepackage{graphicx}
\usepackage{engord}
%\usepackage[english]{babel}
\usepackage{fancyhdr}
\usepackage{amsmath, amssymb}
\usepackage{comment}

\usepackage{listings}

%allows inclusion of url (hyperref is better than url) 
%ref: http://www.fauskes.net/nb/latextips/
\usepackage{hyperref}

%package for chemistry ie: \ce{(NH4)2SO4 -> NH4+ + 2SO4^2-} 
%ref:www.ctan.org/tex-archive/macros/latex/contrib/mhchem/mhchem.pdf
\usepackage[version=3]{mhchem}
%celsius + degrees
\usepackage{gensymb}
%to get last page
\usepackage{lastpage} % \pageref{LastPage}

%make use of the fullpage (no HUGE margins)
\usepackage{fullpage}
\usepackage{subfig}

%allows separating cell in table by diagonal line
\usepackage{slashbox}




%\renewcommand{\chaptername}{Laboratory}
%\setcounter{chapter}{5}

\usepackage{color}
\usepackage[usenames,dvipsnames, table]{xcolor}
% Include this somewhere in your document



\usepackage[absolute]{textpos}

%column  of multi row in tables
\usepackage{multirow}

%to have vertical text in table
\usepackage{rotating}


%%tikz
\usepackage{tikz}
\usetikzlibrary{arrows,calc}
\usepackage{relsize}
\newcommand\LM{\ensuremath{\mathit{LM}}}
\newcommand\IS{\ensuremath{\mathit{IS}}}

%%%%%%% a virer ici!!!!
\begin{comment}
%Fonts and Tweaks for XeLaTeX
\usepackage{fontspec,xltxtra,xunicode}
%\defaultfontfeatures{Mapping=tex-text}
%\setromanfont[Mapping=tex-text]{Hoefler Text}
\setsansfont[Scale=MatchLowercase,Mapping=tex-text]{Gill Sans}

\definecolor{shade}{HTML}{D4D7FE}	%light blue shade
\definecolor{text1}{HTML}{272727}		%text is almost black
\definecolor{headings}{HTML}{173849} 	%dark blue %%%dark red 70111
\definecolor{title}{HTML}{173849} 	%dark blue %%%dark red 70111

\usepackage{titlesec}				%custom \section
\end{comment}







\author{Johan Boissard - 06-304-679}
\date{\today}
\title{MacroEconomics - HWII}
\begin {document}

\maketitle
%\tableofcontents


%%1
\section{The $FE$ curve, capital immobility and monetary policy}
\subsection{ }
If $\kappa\rightarrow0$, the $FE$ curve becomes
\begin{equation}
	\label{eq:FE}
	i = i^{\text{world}} + \frac{E_{+1}^e-E}{E}
\end{equation}
and if $i<i^{\text{world}}$ we have
\begin{equation}
	i-i^{\text{world}} = \frac{E_{+1}^e-E}{E} <0
\end{equation}
and thus in this case
\begin{equation}
	i<i^{\text{world}} \Leftrightarrow E_{+1}^e <E
\end{equation}

\subsection{ }
Capital perfectly immobile means $\kappa = 0$ and exchange rates flexible is $OR=0$ and thus $BP=CA+CP=0$.

Under these circumstances, we can rewrite the original $FE$ curve (eq \ref{eq:FE}) setting $\kappa=0$ which leads to

\begin{equation}
	\label{eq:FEumformed}
	Y = \frac{x_1}{m_1}Y^{\text{world}}+\frac{m_2+x_2}{m_1}R
\end{equation}
and one immediately sees that it is independent from $i$, graphically it is a vertical line in the $i$-$Y$ diagram. 

\subsection{ }
The $LM$ curve reads
\begin{equation}
	i = \frac{k}{h}Y- \frac{1}{h}\overline M 
\end{equation}
so an increase in $\overline M$ will drive up the curve: higher $i$ for a given $Y$.

At this point there is a disequilibrium between the three curves and the $(Y_{FE},i_{FE})$ coordinates at which the $LM$ curve crosses the $FE$ are higher than the coordinates where the $LM$ curve crosses the $IS$ curve $(Y_{IS}, Yi_{is})$:
$\begin{pmatrix}
	Y_{FE}\\
	i_{FE}
\end{pmatrix}
>
\begin{pmatrix}
	Y_{IS}\\
	i_{IS}
\end{pmatrix}$

The only way to get back to an equilibrium is for $R$ to change. 


If $R$ increase the $IS$ curve will increase by the associated multiplier and the $FE$ curve also increases by its associated multiplier. However the multiplier of the $FE$ curve is greater than the $IS$ one (see section \ref{sec:14}). If $R$ increases, the equilibrium will never be reached because $Y_{FE}$, which is initially bigger than $Y_{IS}$, will always grow more than $Y_{IS}$ and thus the only way to reach an equilibrium is to \textbf{decrease} $R$ and thus the $IS$ curve will decrease a little bit and the $FE$ (which grows faster) will eventually catch-up.

To find the new value of $i$, we first find the equilibrium interest rate $i$ in algebraic terms. We can do so in three steps
\begin{enumerate}
	\item Substitute $R$ in the $IS$ curve from the $FE$ curve, see eq \ref{eq:ISFE}
	\item Substitute $Y$ in the $IS$ curve from the $LM$ curve, see eq \ref{eq:ISLM}.
	\item an rearrange eq \ref{eq:rearr}
\end{enumerate}

\begin{eqnarray}
	\label{eq:ISFE}
	i &=& -\frac{1-c+m_1}{b}Y+\frac{x_2+m_2}{b}
	\left(
	\frac{m_1}{m_2+x_2}Y- \frac{x_1}{m_2+x_2}Y^{\text{world}}
	\right) 
	+
	\frac{\overline I + G + x_1Y^{\text{world}}}{b}
	\nonumber\\
	&=&
	-\frac{1-c}{b}Y
	+
	\frac{\overline I + G}{b}
	\\
	\label{eq:ISLM}
	i&=&
	-\underbrace{\frac{1-c}{b}}_{\epsilon>0}
	\left(\frac{ih +\overline M}{k}\right)
	+
	\underbrace{\frac{\overline I + G}{b}}_{\gamma>0}\\
	\label{eq:rearr}
	i &=&
	\frac{1}{1+\epsilon\frac{h}{k}}
	\left(
		\gamma-\epsilon\frac{\overline M}{k}
	\right)
\end{eqnarray}

Since $\gamma, \epsilon, h, k>0$ a decrease in $\overline M$ leads to a decrease in $i$.



\begin{figure}[htbp]
	\centering
		
	

\begin{tikzpicture}[
        scale=2,
        IS/.style={blue, thick},
        LM/.style={red, thick},
        axis/.style={very thick, ->, >=stealth', line join=miter},
        important line/.style={thick}, dashed line/.style={dashed, thin},
        every node/.style={color=black},
        dot/.style={circle,fill=black,minimum size=4pt,inner sep=0pt,
            outer sep=-1pt},
    ]
    % axis
    \draw[axis,<->] (2.5,0) node(xline)[right] {$Y$} -|
                    (0,2.5) node(yline)[above] {$i$};
	
	%FE
	\draw[important line, green, xshift=.1cm]
     (1.2,0) coordinate (es) -- (1.2,2) coordinate (ee)
     node [above right] {$FE$};

	%FE
	\draw[important line, green, xshift=.1cm]
     (.5,0) coordinate (es) -- (.5,2) coordinate (ee)
     node [above right] {$FE'$};
	
	
	%LM
	\draw[important line, red, xshift=.1cm]
	            (.1,.1) coordinate (es) -- (1.7,2) coordinate (ee)
	            node [above right] {$LM'$};
	
	%LM after shift
	\draw[important line, red, xshift=.1cm]		         
	   (.6,.1) coordinate (es) -- (2.3,2) coordinate (ee)
	   node [above right] {$LM$};
	
	%IS
	\draw[important line, blue, xshift=.1cm]
	            (.1,2) coordinate (es) -- (1.7,.2) coordinate (ee)
	            node [above right] {$IS$};
	
	
	%IS
	\draw[important line, blue, xshift=.1cm]
	            (.1,1) coordinate (es) -- (1.1,0) coordinate (ee)
	            node [above right] {$IS'$};


	
	

	
	\node[dot,label=above:$B$] at (.6,.6) (int1) {};
	\node[dot,label=above:$A$] at (1.32,.8) (int1) {};
	
	\draw[->, very thick, black, >=stealth']
       (1.5,1) -- (1.2,1.3)
        node[sloped, above, midway] {$1. \mathsmaller{\Delta M < 0}$};
\end{tikzpicture}
\caption{shift caused by a decrease in Money supply: $IS$ and $FE$ respond by shifting to the left ($\Delta R<0$) ultimately leading to a lower $Y$ and a lower $i$, see text for more explanations.}
\label{fig:ISLM1}
\end{figure}

\subsection{ }
\label{sec:14}
The $FE$-curve reads as in eq \ref{eq:FEumformed} and if we rewrite the $IS$-curve as $Y(i, R)$ we get
\begin{equation}
	Y_{IS} = -\frac{b}{1-c+m_1}i 
	+\frac{m_2+x_2}{1-c+m_1}R
	+\frac{1}{1-c+m_1}(\overline I+G+x_1Y^{\text{world}})
\end{equation}

If we now look at the partial derivative of those two curves (the derivative times the difference in $Y$ is the amount by which the curves will be shifted; multiplier) we get
\begin{eqnarray}
	\frac{\partial Y_{IS}}{\partial R}&=&\frac{m_2+x_2}{1-c+m_1}\\
	\frac{\partial Y_{FE}}{\partial R}&=&\frac{m_2+x_2}{m_1}
\end{eqnarray}
since $c\in(0,1) \Rightarrow(1-c)\in(0,1)$ and $m_1>0$. We have $1-c+m_1>m_1$ and thus 
\begin{equation}
	\frac{\partial Y_{FE}}{\partial R}
	>
	\frac{\partial Y_{IS}}{\partial R}
\end{equation}
which means that the shift in the $FE$ curve is always going to be larger than the shift in the $IS$ curve.


%%% 2
\section{Economic growth and capital markets}


\subsection{ }
According to the Solow model, at equilibrium we have
\begin{equation}
	\Delta K^* = sY -\delta K^* = sF(K^*,L) - \delta K^* = 0
\end{equation}
Setting $F=\sqrt{KL}$ and solving for $K^*$ gives

\begin{equation}
	K^* = \left(\frac{s}{\delta}\right)^2L 
\end{equation}
Since country B does not save ($s=0$), its steady capital is $K_B^*=0$ and $Y_B^*=0$ whereas country A has $s=.25$ and $K_A^*=625$ and $Y_A^*=250$.
Thus the consumptions per capita are
\begin{eqnarray}
	c_A =& \frac{Y_A(1-s_a)}{L} =& \frac{250(1-.25)}{100}=1.875\\
	c_B =& \frac{Y_B(1-s_B)}{L} =& \frac{0(1-0)}{100}=0
\end{eqnarray}

\subsection{ }
%page 282
Since $s_B=0$, all investments will have to be done by country A. 
At equilibrium both saving rates are the same thus $K=K_B=K_B$.

We can write for the world (note the $\frac{1}{2}$ in front of $Y_B$, this is because we take only the capital income in account)
\begin{eqnarray}
	s_a(Y_A(K_A^*)+\frac{1}{2}Y_B(K_B^*))&=&\delta (K_A^*+K_B^*)
	\\
	\frac{3}{2}s_a\sqrt{K^*L} &=& 2\delta K^*
	\\
	K^*&=&\left(\frac{3}{4}\frac{s_a}{\delta}\right)^2L\nonumber\\ 
	&=&\left(\frac{3}{4}\frac{.25}{.1}\right)^2 100%\nonumber\\
	%&=&
	=351.56
\end{eqnarray}

Hence $Y_A=Y_B=10\sqrt{351.56}=187.50$.

\subsection{ }
$GNP$ is the sum of the $GDP$ plus the capital income from abroad minus the capital income generated at home by foreign capital.
\begin{eqnarray}
	GNP_A &=& GDP_A + \frac{1}{2}GDP_B\\
	&=& 187.5 + \frac{1}{2}187.5  = 281.25\\
	GNP_B &=& GDP_B - \frac{1}{2}GDP_A\\
	&=& 187.5  - \frac{1}{2}187.5  = 93.75
\end{eqnarray}

%\subsection{ }
If we define the consumption level, $c$, as the portion of the GDP that makes consumption (ie $c = C/Y$) so that $c=1-s$, we have
\begin{eqnarray}
	c_A &=& 1-s_A = 75\%\\ 
	c_B &=& 1-s_B = 100\%
\end{eqnarray}

\subsection{ }
We see that the sum of the GDPs when there is no trade ($250+0=250$) is smaller as the GDP sum when there is trade ($187.5 +187.5 =375$), so the world aggregate consumption is to be maximized, one has to favor a global capital market. 

When there is a global capital market $K_A$ decreased and since the labour income is defined as half the GDP of A that is in turn increasing with $K$. The higher is $K$ the higher is the labour income, so one does not want to favor global capital market in this case.

\subsection{ }
As argued above, for wealthy countries (in this problem, country A), globalization decrease the capital and consequently the labor income whereas GNP increases.

%%% 3 %%%
\section{Disinflation and oil price shocks}
\subsection{ }
 We have $Y=Y^*$ and thus substituting this into the $SAS$ curve gives
\begin{equation}
	\pi = \underbrace{\pi_{-1}}_{10}+ \underbrace{Y-Y^*}_{0}+10\ln{0.5}
	=3.069\%
\end{equation}


\subsection{ }
\label{sec:2}
If income is to remain at $Y=100$, then the $DAD$ curves becomes ($Y=Y_{-1}$)
\begin{equation}
	\pi = \mu
\end{equation}
Moreover note that the price of oil has fallen \textbf{permanently} to $.5$ and thus
\begin{equation}
	\label{eq:DADOIL}
	\pi_{\text{OIL}, i}=
	\ln{P_{\text{OIL}}, i}-\ln{P_{\text{OIL},i-1}}
	= 0
\end{equation}
for $i>1$.

Which leads to the following $SAS$ curve (for $i>1$)
\begin{equation}
	\pi_i = \pi_{i-1}
\end{equation}

For every period the $DAD$ curve adjusts to the $SAS$ curve which gives
\begin{itemize}
	\item Period 1:
	\begin{eqnarray}
		\pi_1 =& 3.069\%&DAD\\
		\mu_1 =& \pi_1=3.069\%&SAS
	\end{eqnarray}
	\item Period 2:
	\begin{eqnarray}
		\pi_2 =& \pi_1=3.069\%&DAD\\
		\mu_2 =& \pi_2=3.069\%&SAS
	\end{eqnarray}
	\item Period 3:
	\begin{eqnarray}
		\pi_3 =& \pi_2= 3.069\%&DAD\\
		\mu_3 =& \pi_3=3.069\%&SAS
	\end{eqnarray}
	\item ...
	\item Period $n$:
	\begin{eqnarray}
		\pi_n =& \pi_{n-1}= 3.069\%&DAD\\
		\mu_n =& \pi_n=3.069\%&SAS
	\end{eqnarray}
\end{itemize}

\subsection{ }
Here we set $\pi_1=0$, 
and solve
\begin{equation}
	0 = 10 + (Y_1-100)-6.9
\end{equation}
and find $Y_1=96.6$.

To find $\mu$ we solve
\begin{equation}
	0 = \mu -\underbrace{\frac{1}{2}(96.9-100)}_{-1.55}
\end{equation}

For period $i>1$,eq \ref{eq:DADOIL} applies and the $DAD$ curve becomes ($\pi_i=0$)
\begin{equation}
	Y_i=Y^*=100
\end{equation} 
note that for $i>2$, the $DAD$ curve becomes ($Y_i = Y_{i-1}$)
\begin{equation}
	\mu_i=0
\end{equation}

Keeping this in mind we find
\begin{itemize}
	\item Period 1:
	\begin{eqnarray}
		Y_1&=&96.6\\
		\mu_1 &=& .5(Y_1-Y_{-1}) = -1.55
	\end{eqnarray}
	\item Period 2:
	\begin{eqnarray}
		Y_2&=&100\\
		\mu_2 &=& .5(Y_2-Y_{1}) = 1.55
	\end{eqnarray}
	\item Period $i>2$:
	\begin{eqnarray}
		Y_i&=&100\\
		\mu_i &=& .5(Y_i-Y_{i-1}) =0
	\end{eqnarray}
\end{itemize}
\subsection{ }
We have (where $P_{OIL, i}=1$ for $i\neq1$ and $P_{OIL,1}=.5$)
\begin{eqnarray}
	\label{eq:log}
	\pi_{OIL, 1} &= \ln{P_{OIL, 1}}-\ln{P_{OIL, 0}} &=-.69\\
	\pi_{OIL, 2} &= \ln{P_{OIL, 2}}-\ln{P_{OIL, 1}} &=.69\\
	\pi_{OIL, i} &= \ln{P_{OIL, i}}-\ln{P_{OIL, i-1}} &=0 \hspace{7 mm}i >2
\end{eqnarray}

as in \ref{sec:2} $Y_i=Y^*$ for $i\in\mathbb N^+$, we have
\begin{itemize}
	\item Period 1:
	\begin{eqnarray}
		\pi_1 = \pi_{0} + 10\pi_{OIL, 1}=10-6.9=3.1\\
		\mu_1 = \pi_1
	\end{eqnarray}
	\item Period 2:
	\begin{eqnarray}
		\pi_2 = \pi_{1} +  10\pi_{OIL, 2}=3.1+6.9 = 10\\
		\mu_2 = \pi_2
	\end{eqnarray}
	\item Period $i>2$:
	\begin{eqnarray}
		\pi_i = \pi_{i-1} +  10\pi_{OIL, i}=10 + 0 = 10\\
		\mu_i = \pi_i
	\end{eqnarray}
\end{itemize}

\subsection{ }
the sacrifice ratio is defined as
\begin{equation}
	\text{Sacrifice ratio} = \frac{\text{total income loss}}{\text{inflation reduction}}
\end{equation}

if we want to compute this for period $1$ to $3$ the formumla becomes
\begin{equation}
	\label{eq:SR}
	SR = 100\frac{1}{Y^*}\frac{(Y^*-Y_1)+(Y^*-Y_2)+(Y^*-Y_3)}{\pi_0-\pi4}
\end{equation}
For the big-leap we have
\begin{equation}
	SR = \frac{100}{100}\frac{(100-96.9)+(100-100)+(100-100)}{10} = 0.31
\end{equation}

Before calculating the $SR$ when price are reduced for one year only, we first have to calculate the different $Y_i$ and $\mu_i$.

Firstly note that eq \ref{eq:log} is still valid and hence we have

\begin{itemize}
	\item Period 1:
	\begin{eqnarray}
		Y_1 &=& 96.9\\
		\mu_1 &=& .5(Y_1-Y_0) = -1.55
	\end{eqnarray}
	\item Period 2:
	\begin{eqnarray}
		Y_2 &=& 100 -6.9 = 93.1\\
		\mu_2 &=& .5(Y_2-Y_1) = -0.95
	\end{eqnarray}
	\item Period 3:
	\begin{eqnarray}
		Y_3 &=& 100 +0 = 100\\
		\mu_3 &=& .5(Y_3-Y_2) = 3.45
	\end{eqnarray}
	\item Period $i>3$:
	\begin{eqnarray}
		Y_i &=& 100 +0 = 100\\
		\mu_i &=& .5(Y_i-Y_{i-1}) = 0
	\end{eqnarray}
\end{itemize}

Now using \ref{eq:SR}, we find when the oil price is only reduced for one year
\begin{equation}
	SR = \frac{100}{100}\frac{(100-96.9)+(100-93.1)+(100-100)}{10} = 1.
\end{equation}

\end{document}
