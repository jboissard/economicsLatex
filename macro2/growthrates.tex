%
%  untitled
%
%  Created by Johan Boissard [] on 2010-06-24.
%  Copyright (c) Johan Boissard. All rights reserved.
% hhh

\documentclass[a4paper] {scrartcl}
\usepackage[T1]{fontenc}
\usepackage[utf8]{inputenc}
\usepackage{graphicx}
\usepackage{engord}
%\usepackage[english]{babel}
\usepackage{fancyhdr}
\usepackage{amsmath}
\usepackage{comment}

\usepackage{listings}

%allows inclusion of url (hyperref is better than url) 
%ref: http://www.fauskes.net/nb/latextips/
\usepackage{hyperref}

%package for chemistry ie: \ce{(NH4)2SO4 -> NH4+ + 2SO4^2-} 
%ref:www.ctan.org/tex-archive/macros/latex/contrib/mhchem/mhchem.pdf
\usepackage[version=3]{mhchem}
%celsius + degrees
\usepackage{gensymb}
%to get last page
\usepackage{lastpage} % \pageref{LastPage}

%make use of the fullpage (no HUGE margins)
\usepackage{fullpage}
\usepackage{subfig}

%allows separating cell in table by diagonal line
\usepackage{slashbox}




%\renewcommand{\chaptername}{Laboratory}
%\setcounter{chapter}{5}

\usepackage{color}
\usepackage[usenames,dvipsnames, table]{xcolor}
% Include this somewhere in your document



\usepackage[absolute]{textpos}

%column  of multi row in tables
\usepackage{multirow}

%to have vertical text in table
\usepackage{rotating}


%%%%%%% a virer ici!!!!
\begin{comment}
%Fonts and Tweaks for XeLaTeX
\usepackage{fontspec,xltxtra,xunicode}
%\defaultfontfeatures{Mapping=tex-text}
%\setromanfont[Mapping=tex-text]{Hoefler Text}
\setsansfont[Scale=MatchLowercase,Mapping=tex-text]{Gill Sans}

\definecolor{shade}{HTML}{D4D7FE}	%light blue shade
\definecolor{text1}{HTML}{272727}		%text is almost black
\definecolor{headings}{HTML}{173849} 	%dark blue %%%dark red 70111
\definecolor{title}{HTML}{173849} 	%dark blue %%%dark red 70111

\usepackage{titlesec}				%custom \section
\end{comment}







\author{Johan Boissard}
\date{\today}
\title{Growth Rates}
\begin {document}

\maketitle
%\tableofcontent


\section{Definitions of growth rates}
\subsection{Absolute growth rate}
\paragraph{Discrete}
\begin{equation}
	g_{t} = \frac{\Delta x}{\Delta t}
\end{equation}
\paragraph{Continuous} % (fold)
\label{par:continuousa}
\begin{equation}
	g_{t}=\frac{dx}{dt}
\end{equation}
% paragraph continuous (end)
\subsection{Relative growth rate}
\begin{equation}
	R = \frac{x_{t}}{x_{t-1}}
	=
	1+
	\frac{
		g_{1}
	}{
		x_{t-1}
	}
\end{equation}
Note: $\Delta t=1$

\subsection{log growth rate}
This growth rate is the most commonly used in the economic field, most probably because it represents a percentage growth.

\paragraph{Discrete} % (fold)
\label{par:discrete}


\begin{equation}
	r 
	= \log{R} 
	= \log{\left(\frac{x_{t}}{x_{t-1}}\right)}
	\approx
	\frac{x_{t}-x_{t-1}}{x_{t}}
\end{equation}
% paragraph discrete (end)

\paragraph{Continuous} % (fold)
\label{par:continuous}


\begin{equation}
	r 
	= \frac{{\rm d}\log{x}}{{\rm d}t}
	=
	\frac{1}{x}\frac{{\rm d}x}{{\rm d}t}
	=
	\frac{\dot x}{x}
\end{equation}
Note that the following is always true (see section \emph{operator algebra} for more informations)
\begin{equation}
	\frac{{\rm d}y}{y}={\rm d}\ln{(y)}
\end{equation}
% paragraph continuous (end)


\section{Continuously compounded rates}

Usually rates are compounded discretely and have the following form:
\begin{equation}
	\left(1+\frac{r}{k}\right)^{(n\cdot k)}
\end{equation}

Where $r$ is the annual rate, $n$ the number of years and $k$ the frequency (of payment for instance). If the frequency goes towards infinity, we observe the following
\begin{equation}
	\lim_{k\rightarrow\infty}\left(1+\frac{r}{k}\right)^{(n\cdot k)} = e^{rn}
\end{equation} 

\section{Rule of the $70$}

\emph{How many years does it take for an amount to double with rate $r$?}

One can simply answer this question by using the following formula
\begin{equation}
	n = \frac{70}{100}r
\end{equation}

\subsection{Explanation}
The real formula is
\begin{equation}
	(1+r)^n = 2
\end{equation}
which solving for $n$ gives

\begin{equation}
	n = \frac{\log{(2)}}{\log{(1+r)}}
\end{equation}

As Taylor's expansion of $\ln{(x)}$ around $1$ is

\begin{equation}
	\ln{(x)} = \sum_{k=1}^n (-1)^{(k+1)}\frac{(x-1)^k}{k}
\end{equation}

and since $0<r<\frac{1}{10}$ we can keep only the first term, which is $\ln{(x)} =  (x-1)$  and thus $\ln{(1+r)} = r$, which leads to

\begin{equation}
	n \approx \frac{\ln{(2)}}{r} = \frac{.693...}{r}
\end{equation}  


\end{document}
