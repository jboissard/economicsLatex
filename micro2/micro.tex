%
%  untitled
%
%  Created by Johan Boissard [] on 2010-06-24.
%  Copyright (c) Johan Boissard. All rights reserved.
% hhh

\documentclass[a4paper] {scrartcl}
\usepackage[T1]{fontenc}
\usepackage[utf8]{inputenc}
\usepackage{graphicx}
\usepackage{engord}
%\usepackage[english]{babel}
\usepackage{fancyhdr}
\usepackage{amsmath, amssymb}
\usepackage{comment}

\usepackage{listings}

%allows inclusion of url (hyperref is better than url) 
%ref: http://www.fauskes.net/nb/latextips/
\usepackage{hyperref}

%package for chemistry ie: \ce{(NH4)2SO4 -> NH4+ + 2SO4^2-} 
%ref:www.ctan.org/tex-archive/macros/latex/contrib/mhchem/mhchem.pdf
\usepackage[version=3]{mhchem}
%celsius + degrees
\usepackage{gensymb}
%to get last page
\usepackage{lastpage} % \pageref{LastPage}

%make use of the fullpage (no HUGE margins)
\usepackage{fullpage}
\usepackage{subfig}

%allows separating cell in table by diagonal line
\usepackage{slashbox}




%\renewcommand{\chaptername}{Laboratory}
%\setcounter{chapter}{5}

\usepackage{color}
\usepackage[usenames,dvipsnames, table]{xcolor}
% Include this somewhere in your document



\usepackage[absolute]{textpos}

%column  of multi row in tables
\usepackage{multirow}

%to have vertical text in table
\usepackage{rotating}


%%%%%%% a virer ici!!!!
\begin{comment}
%Fonts and Tweaks for XeLaTeX
\usepackage{fontspec,xltxtra,xunicode}
%\defaultfontfeatures{Mapping=tex-text}
%\setromanfont[Mapping=tex-text]{Hoefler Text}
\setsansfont[Scale=MatchLowercase,Mapping=tex-text]{Gill Sans}

\definecolor{shade}{HTML}{D4D7FE}	%light blue shade
\definecolor{text1}{HTML}{272727}		%text is almost black
\definecolor{headings}{HTML}{173849} 	%dark blue %%%dark red 70111
\definecolor{title}{HTML}{173849} 	%dark blue %%%dark red 70111

\usepackage{titlesec}				%custom \section
\end{comment}







\author{Johan Boissard}
\date{\today}
\title{MicroEconomics}
\begin {document}

\maketitle
%\tableofcontents

\section{Introduction}

\paragraph{Two Basic Postulates} % (fold)
\label{par:two_basic_postulates}
\begin{itemize}
	\item Rational Choices
	\item Equilibrium
\end{itemize}
% paragraph two_basic_postulates (end)


\paragraph{Pareto Efficiency} % (fold)
\label{par:pareto_efficiency}
allows no "wasted warfare"
% paragraph pareto_efficiency (end)

\subsection{Comptetitive Equilibrium}
$p_e$
\subsection{Monopoly}
Pareto inefficient $\Leftarrow$ not all appartments are rent

\subsection{Rent control}
Pareto inefficient


\section{Budget Set}
\subsection{Budget Constraints}
bundle $(x_1, ..., x_n)$ affordable at price $\mathbf{p}=(p_1, ..., p_n)$ when
\begin{equation}
	\mathbf{p}\cdot \mathbf{x}=p_1x_1+...+p_nx_n\leq m
\end{equation}
$m$ is the consumer's (disposable) income

\subsubsection{Budget Constraint}
\begin{equation}
	\{(\mathbf{x})|\mathbf{x}\geq0 \text{ and } \mathbf{p}\cdot \mathbf{x}=m\}
\end{equation}

\subsubsection{Budget Set}
\begin{equation}
	B(\mathbf{p}, m)=\{(\mathbf{x})|\mathbf{x}\geq0 \text{ and } \mathbf{p}\cdot \mathbf{x}\leq m\}
\end{equation}


\subsection{Uniform \emph{Ad Valorem} Sales Taxes}
\begin{equation}
	(1+t)\mathbf{p}\cdot \mathbf{x}\leq m
\end{equation}
tax levied at rate $t$.

\section{Preferences and Utility Functions}

\subsection{Preference Relations}
\begin{equation}
	x\succeq y \Leftrightarrow U(x)\geq U(y)
\end{equation}

\subsection{Non-unicity of utility function}
Two utility functions represent the same preference relation if and only if there exists an increasing function $\phi$ such that
\begin{equation}
	U = \phi(V)
\end{equation}


\subsection{Marginal Rate of Substitution}

\begin{equation}
	MRS = -\frac{dx_2}{dx_1}
\end{equation}
rate at which consumer is willing to exchange commodity 2 for commodity 1.

Utility, however remains constant $du=0$, taking the total derivative
\begin{eqnarray}
	du &=& 0\nonumber\\
	&=&\frac{\partial u}{\partial x_1}dx_1+\frac{\partial u}{\partial x_2}dx_2
\end{eqnarray}
Leads to
\begin{equation}
	MRS = -\frac{dx_2}{dx_1} = \frac{\frac{\partial u}{\partial x_1}}{\frac{\partial u}{\partial x_2}}
\end{equation}

\subsection{Cobb-Doublas}
La fonction d'utilité de Cobb-Doublas est la suivante
\begin{equation}
	U (x,y) = x^ay^b
\end{equation}
Par transformation monotone, on a
\begin{equation}
	V(x,y) = a\ln{(x)}+ b \ln{(y)}
\end{equation}

The Marginal utility is
\begin{eqnarray}
	\frac{\partial U}{\partial x_1} &= ax^{a-1}y^b\\ 
	\frac{\partial U}{\partial x_2} &= bx^{a}y^{b-1}
\end{eqnarray}
Hence, the marginal rate of substitution is
\begin{equation}
	MRS = \frac{a}{b}\frac{y}{x}
\end{equation}
Note that the $MRS$ is always the same for any utility function ($U$ or $V$) satisfying the same preference relations.

\section{Consumer Behavior}
\subsection{Consumer Problem}
\begin{equation}
	\max U( \mathbf{x})
\end{equation}
s.t.
\begin{equation}
	\mathbf{p}\cdot \mathbf{x}\leq m
\end{equation}


\subsection{Kuhn-Tucker theorem}
Consider the maximization problem
\begin{equation}
	\max f( \mathbf{x})
\end{equation}
subject to
\begin{equation}
	g_i( \mathbf{x})\leq 0; i=1.. k
\end{equation}
Note that $f, g_i: \mathbb R^n\rightarrow \mathbb R$.

Then for a solution $\mathbf{x^*}$ to the maximization problem there exists Kuhn-Tucker multipliers $\lambda_1,...,\lambda_n\in\mathbb R$ such that

\begin{eqnarray}
	\nabla f( \mathbf{x^*})&=& \sum_{i=1}^k \lambda_i\nabla g_i( \mathbf{x^*})\\
	\lambda_i&\geq&0 \forall i\\
	\lambda &=& 0 \text{ if } g_i( \mathbf{x^*})<0
\end{eqnarray}

\subsection{Perfect Substitutes Case}
\begin{equation}
	MRS = -1
\end{equation}
\subsection{Corner Solutions}
Corner solutions, as their name implies, are solutions that are located in a corner, mathematically
\begin{eqnarray}
	x_i&= \frac{m}{p_1}
	\\
	x_j &= 0 &i\neq j
\end{eqnarray}

\subsection{Maximization problem with Cobb-Doublas}

We wanna solve
\begin{equation}
	\max U = a\ln{(x_1)} + b\ln{(x_2)}
\end{equation}
s.t. $g=p_1x+p_2x_2=m$

We use Kuhn-Tucker
\begin{eqnarray}
	\nabla U &=& \sum \lambda_i\nabla g_i\\
	\begin{pmatrix}
		\frac{a}{x_1^*}\\\frac{b}{x_2^*} 
	\end{pmatrix}
	&=& \lambda_1 
	\begin{pmatrix}
		p_1\\p_2
	\end{pmatrix}
\end{eqnarray}
Hence
\begin{eqnarray}
	x_1^* &= \frac{c}{d}\frac{p_2}{p_1}x_2\\
	&= \frac{c}{c+d}\frac{m}{p_1}\\
	x_2^*&= \frac{d}{c+d}\frac{m}{p_2}
\end{eqnarray}


\section{Dual Approaches to Consumer Behavior}

Two ways of modeling rational consumer behavior
\begin{itemize}
	\item utility maximization
	\item expenditure maximization
\end{itemize}

\subsection{Utility Maximization}
\begin{eqnarray}
	\max U( \mathbf{x}) &\text{s.t.}& \mathbf{p}\cdot \mathbf{x}= m
\end{eqnarray}

\paragraph{Marshallian demand function} % (fold)
\label{par:marshallian_demand_function}
\begin{equation}
	(\mathbf{p}, m)\rightarrow \mathbf{x}(\mathbf{p}, m)
\end{equation}
is called the \textbf{Marshallian demand function}
% paragraph marshallian_demand_function (end)

\subparagraph{Indirect Utility function} % (fold)	
\label{par:indirect_utility_function}
\begin{equation}
	v(\mathbf{p}, m) = U(\mathbf{x}(\mathbf{p}, m))
\end{equation}
% paragraph indirect_utility_function (end)


\subsection{Expenditure Minimization}
\begin{eqnarray}
	\min \mathbf{p}\cdot \mathbf{h} &\text{s.t.}&
	U(\mathbf{h})\geq u
\end{eqnarray}

\paragraph{Hicksian demand function} % (fold)
\label{par:hicksian}
(or compensated demand)
\begin{equation}
	(\mathbf{p}, u)\rightarrow \mathbf{h}(\mathbf{p}, u)
\end{equation}
% paragraph hicksian (end)

\paragraph{Expenditure function} % (fold)
\label{par:expenditure_function}
\begin{equation}
	(\mathbf{p},u)\rightarrow e(\mathbf{p}, u) = \underbrace{\mathbf{p}\cdot \mathbf{h}(\mathbf{p},u)}_{m}
\end{equation}
% paragraph expenditure_function (end)

\subsubsection{Relation between Hicksian and Marshallian demand function}
\begin{equation}
	\mathbf{h}(\mathbf{p}, u)=\mathbf{x}(\mathbf{p}, e(\mathbf{p}, u))
\end{equation}


\subsection{Duality}
The following is true
\begin{itemize}
	\item Utility maximization $\Leftrightarrow$ Expenditure minimization
	\item Utility minimization $\Leftrightarrow$ Expenditure maximization
\end{itemize}

\subsection{Properties of Marshallian demand function}

\begin{itemize}
	\item Homogeneity: $\mathbf{x}(\lambda\mathbf{ p}, \lambda m) = \lambda\mathbf{x}(\mathbf{p}, m)$
	\item Additivity: $\mathbf{p}\cdot \mathbf{x} = m$
	\item Symmetry: see Slutsky
	\item $\mathbf{x}(\mathbf{p}, m)$ is 
	\begin{itemize}
		\item increasing in $m$ for \textbf{normal goods}
		\item decreasing in $m$ for \textbf{inferior goods}
		\item increasing in $\mathbf{p}$ for \textbf{ordinary goods}
		\item decreasing in $\mathbf{p}$ for \textbf{Giffen goods}
	\end{itemize}
	
\end{itemize}


\subsection{Properties of Hicksian demand function}
\begin{itemize}
	\item Homogeneity: $\mathbf{h}(\lambda \mathbf{p}, u) = \lambda \mathbf{h}(\mathbf{p},u)$
	\item $U(\mathbf{h}(\mathbf{p},u))=u$
	\item Symmetry: see Slutsky
	\item Monotonicity: $\mathbf{h}(\mathbf{p}, u)$ is decreasing in $\mathbf{p}$
\end{itemize}


\subsection{Roy's Identity}
For any good $i$ we have
\begin{eqnarray}
	x_i(\mathbf{p}, m) = -\frac{\frac{\partial v(\mathbf{p},m)}{\partial p_i}}{\frac{\partial v(\mathbf{p},m)}{\partial m}}
\end{eqnarray}

\begin{eqnarray}
	h_i(\mathbf{p},u) = \frac{\partial e(\mathbf{p},u)}{\partial p_i}
\end{eqnarray}


\subsection{Slutsky matrix}
\begin{equation}
	S = (S_{ij}) = \left(\frac{\partial h_i}{\partial p_j}\right)
\end{equation}
Using Roy's identity, one can write
\begin{equation}
	(S_{ij}) = \left(\frac{\partial^2 e(\mathbf{p},u)}{\partial p_i\partial p_j}\right)
\end{equation}

\begin{equation}
	S\cdot \mathbf{p} = \sum_j S_{ij}p_j = \sum_j \frac{\partial h_i}{\partial p_j}p_j = 0
\end{equation}

\subsection{Slutsky Relations}
\begin{eqnarray}
	S_{ij} &=& \frac{\partial h_i}{\partial p_j}\\
	&=& \frac{\partial x_i}{\partial p_j}
	+\frac{\partial e}{\partial p_j}\frac{\partial x_i}{\partial e}\\
	&=& \frac{\partial x_i}{\partial p_j}
	+h_j\frac{\partial x_i}{\partial e}\\
	&=& \frac{\partial x_i}{\partial p_j}
	+x_j\frac{\partial x_i}{\partial m}
\end{eqnarray}

and therefore we get the Slutsky relations

\begin{equation}
	\frac{\partial x_i}{\partial p_j}
	+x_j\frac{\partial x_i}{\partial m}
	=
	\frac{\partial x_j}{\partial p_i}
	+x_i\frac{\partial x_j}{\partial m}
\end{equation}

\subsubsection{Interpretation}
\begin{equation}
	\frac{\partial x_i}{\partial p_j}
	+x_j\frac{\partial x_i}{\partial m} = S_{ij} = \frac{\partial h_i}{\partial p_j}
\end{equation}

thus
\begin{equation}
	\frac{\partial x_i}{\partial p_j} = 
	\underbrace{\frac{\partial h_i}{\partial p_j}}_{\text{substitution effect}}
	+
	\underbrace{- x_j\frac{\partial x_i}{\partial m}}_{\text{income effect}}
\end{equation}


\section{Labor Supply and Intemporal Choice}


\subsection{Present Values}
\begin{equation}
	c_1+c_2\frac{1}{1+r}=y_1+(1+r)y_2
\end{equation}
where $c_i$ is consumption at time $t$, $y_i$ amount spent and $r$ real interest rate $(1+r)=\frac{1+R}{1+i}$.

Consumer at time $1$ can either save or borrow for/on period $2$.

\begin{itemize}
	\item if $r$ increase, consumer \textbf{saves} more
	\item if $r$ decrease, consumer \textbf{borrows} more
\end{itemize}

\subsection{The $N$-period case}
\begin{eqnarray}
	\max U(c_1, c_2, ..., c_N) \\
	 \text{s.t. }\\
	\sum  \frac{1}{(1+r)^{i-1}}c_i = \sum \frac{1}{(1+r)^{i-1}}y_i
\end{eqnarray}


\section{Choice under Uncertainty}
if we set: 
\begin{itemize}
	\item $c_a$: car accident
	\item $c_{na}$ no car accident
\end{itemize}
$\pi_a$ and $\pi_{na}$ are the associated probabilities, in case of accident L\$ are lost.
\begin{eqnarray}
	c_{na} =& m\\
	c_a =& m-L
\end{eqnarray}
after buying premium $\gamma K$

\begin{eqnarray}
	c_{na} =& m-\gamma K\\
	c_a =& m-L -\gamma K + K = m-L+(1-\gamma)K
\end{eqnarray}
thus $K=\frac{1}{1-\gamma}(c_a-m+L)$ and
\begin{equation}
	c_{na} = \frac{m-\gamma L}{1-\gamma}-\frac{\gamma}{1-\gamma}c_a
\end{equation}


\subsection{competitive insurance (fair)}
expected economic profit is zero
\begin{equation}
	\gamma K -\pi_aK-(1-\pi_a)0=(\gamma-\pi_a)K=0
\end{equation}
\begin{equation}
	\gamma = \pi_a
\end{equation}

The rational choice must satisfy
\begin{equation}
	\frac{\gamma}{1-\gamma}=\frac{\pi_a}{1-\pi_a}=
	\frac{\pi_a}{\underbrace{\pi_{na}}_{1-\pi_a}}\frac{MU(c_a)}{MU(c_{na})} 
\end{equation}
and thus
\begin{equation}
	MU(c_a) = MU(c_{na})
\end{equation}

\section{Market Demand}
The consumer $i$'s ordinary demand function for commodity $j$ is 
\begin{equation}
	x_j^{*i}(\mathbf{p}, m)
\end{equation}
\subsection{Aggregate Demand}
When all consumers are price takers, the market demand for commodity $j$ is
\begin{equation}
	X_j(\mathbf{p}, \mathbf{m}) = \sum_{i=1}^n x_j^{*i}(\mathbf{p}, m^i)
\end{equation}

If all consumers are identical then
\begin{equation}
	X_j(\mathbf{p}, M) = n\cdot x_j^*(\mathbf{p}, m)
\end{equation}
where $M=nm$ and $x^{*i}=x^*$ $\forall i$

\subsection{Elasticity}
\begin{equation}
	\epsilon_{x,y} 
	= \frac{\%\Delta x}{\%\Delta y} 
	= \frac{\partial x}{\partial y}\frac{y}{x}
	=\left |\frac{\partial \ln{(x)}}{\partial\ln{(y)}}\right |
\end{equation}

\subsection{Revenue and Own-Price Elasticity of demand}
The seller's revenue is
\begin{equation}
	R(p) = p\cdot X^{*}(p)
\end{equation}

Variation around price is
\begin{equation}
	\frac{dR}{dp} = X^*(p)+p\frac{dX^*}{dp} = X^*(p)(1+\epsilon)
\end{equation}


\begin{itemize}
	\item $\epsilon\in(-1,0)$, price rise cause revenue to rise
	\item $\epsilon=-1$, price rise has no effect on revenue
	\item $\epsilon<-1$ price rise causes revenue to fall
\end{itemize}

\subsection{Marginal Revenue}
\begin{equation}
	MR(q) = \frac{dR(q)}{dq} = p(q)\cdot\left(1+\frac{1}{\epsilon}\right)
\end{equation}

\section{The Producer}
\subsection{Technology}
A technology is a process by which inputs are converted to an output.


\paragraph{Input bundle} % (fold)
\label{par:input_bundle}
\begin{equation}
	\mathbf{x}=(x_1, x_2, ..., x_n)
\end{equation}
% paragraph input_bundle (end)

\subsection{Production Functions}
States the \textbf{maximum} amount of output possible from the input bundle
\begin{equation}
	y = f(x_1, ..., x_n)
\end{equation}
where $y$ is the level of output.


\paragraph{Technology sets} % (fold)
\label{par:technology_sets}
A \textbf{production plan} is an input bundle and an output level:$(\mathbf{x}, y)$.
A production plan is feasible if 
\begin{equation}
	y\leq f(\mathbf{x})
\end{equation}

The collection of all feasible production plan is called technology set.

The technology set is
\begin{equation}
	T = \{(\mathbf{x}, y)|y\leq f(\mathbf{x})\text{ and } \mathbf{x}\geq0\}
\end{equation}
% paragraph technology_sets (end)

\subsubsection{Example of technologies}
\begin{itemize}
	\item Cobb-Douglas Production function
	\begin{equation}
		y = A\prod_{i=1}^n x_i^{a_i}
	\end{equation}
	\item Fixed proportions (perfect complements)
	\begin{equation}
		y = \min(a_1x_1, ..., a_nx_n)
	\end{equation}
	\item Perfect substitutes
	\begin{equation}
		y = \sum_{i=1}^na_ix_i
	\end{equation}
\end{itemize}

\subsection{Marginal Product}
\begin{equation}
	MP_i = \frac{\partial y}{\partial x_i}
\end{equation}

\subsection{Return-to-scale}
\begin{equation}
	f(k \mathbf{x})=kf(\mathbf{x})
\end{equation}

\begin{itemize}
	\item Diminishing return-to-scale
	\begin{equation}
		f(k \mathbf{x})<kf(\mathbf{x})
	\end{equation}
	\item Increasing return-to-scale
	\begin{equation}
		f(k \mathbf{x})>kf(\mathbf{x})
	\end{equation}
\end{itemize}

\subsection{Technical rate of substitution}
\begin{equation}
	\frac{dx_2}{dx_1} = \frac{\partial y/\partial x_1}{\partial y/\partial x_2}
\end{equation}

\subsection{Well-behaved technologies}
is \textbf{monotonic} and \textbf{convex}

\paragraph{Monotonic} % (fold)
\label{par:monotonic}
more of any input generates more output
% paragraph monotonic (end)

\section{Cost Minimization}
\subsection{Cost minimization problem}
\begin{equation}
	\min w_1x_1+w_2x_2
\end{equation}
subject to $f(x_1,x_2)\geq y$

\paragraph{Iso-cost lines} % (fold)
\label{par:iso_cost_lines}
\begin{equation}
	x_1x_1+w_2x_2 = C
\end{equation}
% paragraph iso_cost_lines (end)


\paragraph{Isoquant} % (fold)
\label{par:isoquant}
\begin{equation}
	f(x_1,x_2)=C
\end{equation}
% paragraph isoquant (end)


\paragraph{Firm's conditional demand for input 1} % (fold)
\label{par:firm_s_conditional_demand_for_input_1}
\begin{equation}
	x_1^*(w_1, w_2, y)
\end{equation}
% paragraph firm_s_conditional_demand_for_input_1 (end)

\paragraph{Firm's total cost function} % (fold)
\label{par:firm_s_total_cost_function}
\begin{equation}
	c(w_1, w_2, y)
\end{equation}
% paragraph firm_s_total_cost_function (end)

\paragraph{Average total cost} % (fold)
\label{par:average_total_cost}
\begin{equation}
	AC = \frac{c(w_1,w_2,y)}{y}
\end{equation}
% paragraph average_total_cost (end)

\subsection{Short-run and long-run}
The short-run total cost minimization problem, is the long-term problem with the additional constraint:
\begin{equation}
	x_i = x_i' = C
\end{equation}
(one input can't be changed)

\section{Profit Maximization}
\subsection{Economic Profit}
\begin{equation}
	\Pi = \sum p_ix_i-w_ix_i
\end{equation}

\paragraph{Present value of firm} % (fold)
\label{par:present_value_of_firm}
\begin{equation}
	PV = \sum \frac{1}{(1+r)^i}\Pi_i
\end{equation}
% paragraph present_value_of_firm (end)

\subsection{Profit maximization problem}
assuming one output, $y$
\begin{equation}
	\max py -\sum w_ix_i
\end{equation}
st $f(\mathbf{x})\geq = y$,  solving with the lagrangian gives
\begin{equation}
	\frac{\partial f}{\partial x_i}=\frac{w_i}{p}
\end{equation}

\subsection{Hotelling's Lemma}
\begin{eqnarray}
	y^*(p,w) &=&\frac{\partial \Pi}{\partial p}(p,w)\\
	x_i^* &=& -\frac{\partial\Pi}{\partial w_i}(p,w)
\end{eqnarray}
similar as Roy's identity, for consumer's problem

\subsection{Profit maximization and marginal cost}
we have
\begin{equation}
	py - C(y,w)
\end{equation}
and thus
\begin{equation}
	\frac{\partial C}{\partial y}=p
\end{equation}

With constant return to scale, profit maximization gives $\Pi=0$

\subsection{Revealed profitability}
The firm's technology set must lie under all the iso profit lines

\section{Monopoly}
\subsection{Pure Monopoly}
\begin{itemize}
	\item Single seller
	\item alter market price by altering output level
\end{itemize}
\textbf{Pareto ineficient}

\subsection{Profit Maximization}
\begin{equation}
	\Pi(y) = p(y)y - c(y)
\end{equation}
where $p(y)$ is the inverse demand function.

\begin{eqnarray}
	\max_y \Pi &\Leftrightarrow&\\
	\frac{d \Pi}{d y} &=& 0\\
	&\Leftrightarrow&\\
	MR &=& MC\\
	p'y^*+p &=& c'\\
	 p(y^*)(1+\frac{dp}{dy}\frac{y^*}{p}) &=&\frac{dc}{dy}\\
	 p(y^*)(1+\frac{1}{\epsilon}) &=&\frac{dc}{dy}
\end{eqnarray}
where $\epsilon$ is the price-owner elasticity.

\subsubsection{Markup pricing}
Markup is the difference between the price and the marginal cost, that make up the profit
\begin{equation}
	p(y^*) - MC
\end{equation}

\subsection{Tax Levied on Monopolist}
\subsubsection{Profit Tax}
Monopolist has now to solve
\begin{equation}
\max_y	(1-t)\Pi(y) \Leftrightarrow \max_y \Pi(y)
\end{equation}
thus, \textbf{neutral tax}.

\subsubsection{Quantity Tax}

\begin{equation}
	MC' = MC +t
\end{equation}
\begin{itemize}
	\item higher price
	\item less quantity
\end{itemize}

called \textbf{distortionary tax}


\subsection{Natural Monopoly}
Arises when firm's technology has economies-of-scale high enough to supply whole market.

Can't set $MC=p$ because it implies $ATC>p \Rightarrow$ economic loss.

\subsection{Price discrimnation}

\subsubsection{First Degree}
Each output sold at different price.

Pareto Efficient.

\subsubsection{Third degree}
Supply different markets with different demands


\section{Oligopoly}
Oligopoly is an industry consisting of a few firms. Each firm's own price or output decisions affect its competitors profit.

\subsection{Duopoly}
\begin{equation}
	\Pi_1(y_1|y_2) = p(y_1+y_2)y_1-c_1(y_1)
\end{equation}
Maximizing:
\begin{equation}
	\frac{\partial \Pi_1}{\partial y_1}=
	p(y_1+y_2) + y_1\frac{\partial p(y_1+y_2)}{\partial y_1}
	- \frac{dc}{dy_1}=0
\end{equation}
Solution is $y_1=R(y_2)$, same story for $\Pi_2$ that leads $y_2 = R(y_1)$.

The solution $y_1^*=R(y_2^*)$ and $y_2^*=R(y_1^*)$ is called the \textbf{Cournot-Nash equilibrium} (quantity competition)


\subsection{Bertrand Games}
Price competitio, equilibrium is $p=p_1=p_2 = MC$.

\end{document}

